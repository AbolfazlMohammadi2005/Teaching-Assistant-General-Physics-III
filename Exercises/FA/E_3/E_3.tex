%---------------------------------------------------
% کلاس سند: article برای اسناد متنی، با اندازه کاغذ A4 و فونت پایه 16
\documentclass[a4paper,16pt]{article}

% بسته‌های ریاضی برای معادلات و نمادهای پیشرفته
\usepackage{amsmath, amssymb, amstext}

% بسته رنگ برای استفاده از رنگ در متن یا محیط‌ها
\usepackage{xcolor}
\usepackage{setspace}

% بسته گرافیک برای درج تصاویر
\usepackage{graphicx}

% تنظیمات حاشیه صفحه
\usepackage{geometry}

% بسته لینک‌دهی (هایپرلینک‌ها)
\usepackage{hyperref}

% بسته صفحه‌آرایی برای تنظیم هدر و فوتر
\usepackage{fancyhdr}


% بسته xepersian برای پشتیبانی از زبان فارسی با XeLaTeX
\usepackage{xepersian}

% تعیین فونت فارسی (فایل فونت باید موجود باشد یا در سیستم نصب باشد)
\settextfont{B Nazanin}

% تنظیم اندازه حاشیه‌ها از هر طرف برابر با 1 اینچ
\geometry{margin=1in}

% تنظیم سبک fancy برای سربرگ و پاورقی
\pagestyle{fancy}

% سمت چپ هدر: مشخص‌کننده سری تمرین و نام درس
\lhead{تمرین سری سه درس فیزیک\lr{ III} }

% وسط هدر: نام دانشکده
\chead{دانشکده فیزیک دانشگاه صنعتی خواجه نصیرالدین طوسی}

% سمت راست هدر: تاریخ روز به‌صورت خودکار
\rhead{\today}

% پایین صفحه: شماره صفحه
\cfoot{\thepage}

% شروع سند
\setstretch{1.3} % فاصله 1.3 برابر فاصله معمولی

\begin{document}
	
	% صفحه عنوان (بدون هدر و فوتر)
	\thispagestyle{empty}
	\begin{center}
		% لوگوی دانشگاه (باید فایل تصویر در مسیر پروژه باشد)
		
		\includegraphics[width=0.5\textwidth]{../../../images/image-E_3/K._N._Toosi_University_of_Technology.png} \\[10pt] 
		% عنوان سند
		\textbf{\LARGE عنوان:تمرین سری سه}\\[20pt]
		
		% نیم‌سال تحصیلی
		\textbf{\LARGE نیم‌سال تحصیلی:4041 }\\[10pt]
		
		% نام مدرس درس
		\textbf{\Large مدرس: دکتر امین نصیری‌راد}\\[10pt]
		
		% مبحث تمرین 
		\textbf{\Large  مبحث تمرین:دما و گرما  }\\[10pt]
		
		% مهلت تحویل تمرین
		\textbf{\Large مهلت تحویل:17آبان }
	\end{center}
	
	% صفحه جدید برای ادامه سند
	\newpage
	
	% ایجاد فهرست مطالب به‌صورت خودکار (براساس \section و \subsection و ...)
	\tableofcontents
	\newpage
	
	
	\section{سوال اول}
	
	فرض کنید دمای یک گاز برابر با $373.15~\mathrm{K}$ باشد، زمانی که در نقطه جوش آب قرار دارد. نسبت فشار گاز در نقطه جوش به فشار آن در نقطه سه‌گانه آب چه مقدار حدی دارد؟  
	(فرض کنید حجم گاز در هر دو دما یکسان باشد.)
	
	
	\section{سوال دوم}
	آب با س 
	در چه دمایی دمای درج شده در مقیاس فارنهایت برابر است با:  
	(الف) دو برابر دمای درج شده در مقیاس سلسیوس و  
	(ب) نصف دمای درج شده در مقیاس سلسیوس؟
	
	\section{سوال سوم}
	 
	فرض کنید روی یک مقیاس دمای خطی $X$، آب در دمای $-53.5^\circ X$ می‌جوشد و در $-170^\circ X$ یخ می‌زند. دمای $340~\mathrm{K}$ روی مقیاس $X$ چند درجه است؟  
	(نقطه جوش آب را تقریباً $373~\mathrm{K}$ در نظر بگیرید.)
	
	\section{سوال چهارم}
	
	یک سوراخ دایره‌ای در یک صفحه آلومینیومی در دمای $0.0~^\circ\mathrm{C}$، قطر $2.725~\mathrm{cm}$ دارد. قطر آن هنگامی که دمای صفحه به $100.0~^\circ\mathrm{C}$ افزایش می‌یابد، چقدر خواهد شد؟
	
	
	\section{سوال پنجم}
	  
	یک میله پرچم آلومینیومی به ارتفاع $33~\mathrm{m}$ دارد. طول آن با افزایش دما به اندازه $15~^\circ\mathrm{C}$، چقدر افزایش می‌یابد؟
	\section{سوال ششم}
	
	یک فنجان آلومینیومی با ظرفیت $100~\mathrm{cm^3}$ به طور کامل با گلیسیرین در دمای $22~^\circ\mathrm{C}$ پر شده است. اگر دمای هر دو فنجان و گلیسیرین به $28~^\circ\mathrm{C}$ افزایش یابد، چه مقدار گلیسیرین (در صورت وجود) از فنجان سرریز خواهد شد؟  
	(ضریب انبساط حجمی گلیسیرین $\beta = 5.1 \times 10^{-4}~^\circ\mathrm{C}^{-1}$ است.)
	
	\section{سوال هفتم}
	 
	به دنبال افزایش دمای $32~^\circ\mathrm{C}$، میله‌ای که در مرکز آن ترک دارد، به سمت بالا کج می‌شود.  
	فاصله ثابت $L_0 = 3.77~\mathrm{m}$ و ضریب انبساط خطی میله $\alpha = 25 \times 10^{-6}~^\circ\mathrm{C}^{-1}$ است.  
	ارتفاع $x$ مرکز میله را بیابید.
	\begin{center}
		\includegraphics[width=0.5\textwidth]{../../../images/image-E_3/Image-1.png} \\[10pt] 
	\end{center}
		\section{سوال هشتم}
		  
		پس از انتقال $50.2~\mathrm{kJ}$ انرژی به عنوان گرما به $260~\mathrm{g}$ آب مایع که در نقطه انجماد خود قرار دارد، چه مقدار آب هنوز منجمد نشده باقی می‌ماند؟
	\section{سوال نهم}
	 
	(الف) دو قالب یخ به جرم $50~\mathrm{g}$ هر کدام، در $200~\mathrm{g}$ آب در یک ظرف عایق حرارتی ریخته می‌شوند. اگر دمای اولیه آب $25~^\circ\mathrm{C}$ باشد و یخ‌ها مستقیماً از فریزر با دمای $-15~^\circ\mathrm{C}$ آمده باشند، دمای نهایی در تعادل حرارتی چقدر خواهد بود؟  
	
	(ب) اگر فقط یک قالب یخ استفاده شود، دمای نهایی چه مقدار خواهد بود؟
	\\
	\vspace{10pt}
	\textbf{موفق باشید.}
\end{document}