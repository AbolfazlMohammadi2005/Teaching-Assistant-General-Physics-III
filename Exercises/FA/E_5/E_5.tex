%---------------------------------------------------
% کلاس سند: article برای اسناد متنی، با اندازه کاغذ A4 و فونت پایه 16
\documentclass[a4paper,16pt]{article}

% بسته‌های ریاضی برای معادلات و نمادهای پیشرفته
\usepackage{amsmath, amssymb, amstext}

% بسته رنگ برای استفاده از رنگ در متن یا محیط‌ها
\usepackage{xcolor}
\usepackage{setspace}

% بسته گرافیک برای درج تصاویر
\usepackage{graphicx}

% تنظیمات حاشیه صفحه
\usepackage{geometry}

% بسته لینک‌دهی (هایپرلینک‌ها)
\usepackage{hyperref}

% بسته صفحه‌آرایی برای تنظیم هدر و فوتر
\usepackage{fancyhdr}


% بسته xepersian برای پشتیبانی از زبان فارسی با XeLaTeX
\usepackage{xepersian}

% تعیین فونت فارسی (فایل فونت باید موجود باشد یا در سیستم نصب باشد)
\settextfont{B Nazanin}

% تنظیم اندازه حاشیه‌ها از هر طرف برابر با 1 اینچ
\geometry{margin=1in}

% تنظیم سبک fancy برای سربرگ و پاورقی
\pagestyle{fancy}

% سمت چپ هدر: مشخص‌کننده سری تمرین و نام درس
\lhead{تمرین سری پنج درس فیزیک\lr{ III} }

% وسط هدر: نام دانشکده
\chead{دانشکده فیزیک دانشگاه صنعتی خواجه نصیرالدین طوسی}

% سمت راست هدر: تاریخ روز به‌صورت خودکار
\rhead{\today}

% پایین صفحه: شماره صفحه
\cfoot{\thepage}

% شروع سند
\setstretch{1.3} % فاصله 1.3 برابر فاصله معمولی

\begin{document}
	
	% صفحه عنوان (بدون هدر و فوتر)
	\thispagestyle{empty}
	\begin{center}
		% لوگوی دانشگاه (باید فایل تصویر در مسیر پروژه باشد)
		
		\includegraphics[width=0.5\textwidth]{../../../images/image-E_5/K._N._Toosi_University_of_Technology.png} \\[10pt] 
		% عنوان سند
		\textbf{\LARGE عنوان:تمرین سری‌پنج}\\[20pt]
		
		% نیم‌سال تحصیلی
		\textbf{\LARGE نیم‌سال تحصیلی:4041 }\\[10pt]
		
		% نام مدرس درس
		\textbf{\Large مدرس: دکتر امین نصیری‌راد}\\[10pt]
		
		% مبحث تمرین 
		\textbf{\Large  مبحث تمرین:نظریه جنبشی گازها  }\\[10pt]
		
		% مهلت تحویل تمرین
		\textbf{\Large مهلت تحویل:۱۱آذر }
	\end{center}
	
	% صفحه جدید برای ادامه سند
	\newpage
	
	% ایجاد فهرست مطالب به‌صورت خودکار (براساس \section و \subsection و ...)
	\tableofcontents
	\newpage
	
	
	\section{سوال اول}
	 
	جرم به کیلوگرم $7.50 \times 10^{24}$ اتم آرسنیک را بیابید، که جرم مولی آن $74.9~\mathrm{g/mol}$ است.
	\section{سوال دوم}
	  
	یک ظرف شامل $2~\mathrm{mol}$ گاز ایده‌آل با جرم مولی $M_1$ و $0.5~\mathrm{mol}$ گاز ایده‌آل دوم با جرم مولی $M_2 = 3 M_1$ است.  
	چه کسری از فشار کل روی دیواره ظرف به گاز دوم تعلق دارد؟  
	(توضیح نظریه جنبشی فشار منجر به قانون تجربی فشار جزئی برای مخلوطی از گازها می‌شود که واکنش شیمیایی با هم ندارند: فشار کل وارد بر مخلوط برابر با مجموع فشارهایی است که هر گاز جداگانه اگر کل ظرف را اشغال کند، اعمال می‌کند.)
	
	\section{سوال سوم}

ظرف \lr{A }در شکل زیر شامل یک گاز ایده‌آل با فشار $5.0 \times 10^5~\mathrm{Pa}$ و دمای $300~\mathrm{K}$ است. این ظرف توسط یک لوله نازک (و یک شیر بسته) به ظرف \lr{B} متصل شده است، که حجم آن چهار برابر ظرف \lr{A} است.  

ظرف \lr{B }همان گاز ایده‌آل را با فشار $1.0 \times 10^5~\mathrm{Pa}$ و دمای $400~\mathrm{K}$ در خود دارد.  
شیر باز می‌شود تا فشارها برابر شوند، اما دمای هر ظرف حفظ می‌شود.  
در این صورت فشار نهایی چقدر خواهد بود؟
	\begin{center}
		\includegraphics[width=0.5\textwidth]{../../../images/image-E_5/Image-1.png} \\[10pt] 
	\end{center}
	
	\section{سوال چهارم}
 
یک پالس از مولکول‌های هیدروژن $(\mathrm{H_2})$ به سمت دیوار پرتاب می‌شود، به طوری که زاویه آن با عمود دیوار $55^\circ$ است.  
هر مولکول در پالس دارای سرعت $1.0~\mathrm{km/s}$ و جرم $3.3 \times 10^{-24}~\mathrm{g}$ است.  
پالس با نرخ $10^{23}$ مولکول در ثانیه، روی سطحی به مساحت $2.0~\mathrm{cm^2}$ به دیوار برخورد می‌کند.  
فشار پالس بر دیوار چقدر است؟
	\section{سوال پنجم}
 
انرژی جنبشی انتقالی متوسط مولکول‌های نیتروژن در دمای $1600~\mathrm{K}$ چقدر است؟
	\section{سوال ششم}
	 
	در یک شتاب‌دهنده ذرات خاص، پروتون‌ها در مسیر دایره‌ای به قطر $23.0~\mathrm{m}$ در یک محفظه خلأ حرکت می‌کنند، که گاز باقیمانده در آن دما $295~\mathrm{K}$ و فشار $1.00 \times 10^{-6}~\mathrm{torr}$ دارد.  
	
	(الف) تعداد مولکول‌های گاز در هر سانتی‌متر مکعب در این فشار چقدر است؟ 
	\\ 
	(ب) مسیر آزاد متوسط مولکول‌های گاز اگر قطر مولکولی $2.00 \times 10^{-8}~\mathrm{cm}$ باشد، چقدر است؟
	\newpage
	\section{سوال هفتم}
	
	یک مولکول هیدروژن (قطر $1.0 \times 10^{-8}~\mathrm{cm}$) که با سرعت \lr{RMS }حرکت می‌کند، از یک کوره با دمای $4000~\mathrm{K}$ به محفظه‌ای وارد می‌شود که شامل اتم‌های آرگون سرد (قطر $3.0 \times 10^{-8}~\mathrm{cm}$) با چگالی $4.0 \times 10^{19}~\mathrm{atoms/cm^3}$ است.  
	
	(الف) سرعت مولکول هیدروژن چقدر است؟  
	(ب) اگر با یک اتم آرگون برخورد کند، نزدیک‌ترین فاصله بین مراکز آن‌ها چقدر است، با در نظر گرفتن هر کدام به شکل کروی؟  
	(ج) تعداد اولیه برخوردها در ثانیه که مولکول هیدروژن تجربه می‌کند، چقدر است؟
	\vspace{10pt}
	\\
	\textbf{موفق باشید.}
\end{document}