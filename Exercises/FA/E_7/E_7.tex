%---------------------------------------------------
% کلاس سند: article برای اسناد متنی، با اندازه کاغذ A4 و فونت پایه 16
\documentclass[a4paper,16pt]{article}

% بسته‌های ریاضی برای معادلات و نمادهای پیشرفته
\usepackage{amsmath, amssymb, amstext}

% بسته رنگ برای استفاده از رنگ در متن یا محیط‌ها
\usepackage{xcolor}
\usepackage{setspace}

% بسته گرافیک برای درج تصاویر
\usepackage{graphicx}

% تنظیمات حاشیه صفحه
\usepackage{geometry}

% بسته لینک‌دهی (هایپرلینک‌ها)
\usepackage{hyperref}

% بسته صفحه‌آرایی برای تنظیم هدر و فوتر
\usepackage{fancyhdr}


% بسته xepersian برای پشتیبانی از زبان فارسی با XeLaTeX
\usepackage{xepersian}

% تعیین فونت فارسی (فایل فونت باید موجود باشد یا در سیستم نصب باشد)
\settextfont{B Nazanin}

% تنظیم اندازه حاشیه‌ها از هر طرف برابر با 1 اینچ
\geometry{margin=1in}

% تنظیم سبک fancy برای سربرگ و پاورقی
\pagestyle{fancy}

% سمت چپ هدر: مشخص‌کننده سری تمرین و نام درس
\lhead{تمرین سری هفت درس فیزیک\lr{ III} }

% وسط هدر: نام دانشکده
\chead{دانشکده فیزیک دانشگاه صنعتی خواجه نصیرالدین طوسی}

% سمت راست هدر: تاریخ روز به‌صورت خودکار
\rhead{\today}

% پایین صفحه: شماره صفحه
\cfoot{\thepage}

% شروع سند
\setstretch{1.3} % فاصله 1.3 برابر فاصله معمولی

\begin{document}
	
	% صفحه عنوان (بدون هدر و فوتر)
	\thispagestyle{empty}
	\begin{center}
		% لوگوی دانشگاه (باید فایل تصویر در مسیر پروژه باشد)
		
		\includegraphics[width=0.5\textwidth]{../../../images/image-E_7/K._N._Toosi_University_of_Technology.png} \\[10pt] 
		% عنوان سند
		\textbf{\LARGE عنوان:تمرین سری هفت}\\[20pt]
		
		% نیم‌سال تحصیلی
		\textbf{\LARGE نیم‌سال تحصیلی:4041 }\\[10pt]
		
		% نام مدرس درس
		\textbf{\Large مدرس: دکتر امین نصیری‌راد}\\[10pt]
		
		% مبحث تمرین 
		\textbf{\Large  مبحث تمرین:آنتروپی  }\\[10pt]
		
		% مهلت تحویل تمرین
		\textbf{\Large مهلت تحویل:۷دی }
	\end{center}
	
	% صفحه جدید برای ادامه سند
	\newpage
	
	% ایجاد فهرست مطالب به‌صورت خودکار (براساس \section و \subsection و ...)
	\tableofcontents
	\newpage
	
	
	\section{سوال اول}
	
	 
	یک نمونه $2.50~\mathrm{mol}$ از یک گاز ایده‌آل به صورت برگشت‌پذیر و ایزوترمیک در دمای $360~\mathrm{K}$ منبسط می‌شود تا حجم آن دو برابر شود.  
	افزایش آنتروپی گاز چقدر است؟
	\section{سوال دوم}
	
	
	(الف) تغییر آنتروپی یک قالب یخ $12.0~\mathrm{g}$ که به طور کامل در سطل آبی که دمای آن کمی بالاتر از نقطه انجماد آب است ذوب می‌شود، چقدر است؟  
	\\
	(ب) تغییر آنتروپی یک قاشق $5.00~\mathrm{g}$ آب که به طور کامل روی صفحه گرمی که دمای آن کمی بالاتر از نقطه جوش آب است تبخیر می‌شود، چقدر است؟
	\section{سوال سوم}
	 
	یک قالب یخ $10~\mathrm{g}$ با دمای $-10~^\circ\mathrm{C}$ در دریاچه‌ای با دمای $15~^\circ\mathrm{C}$ قرار داده می‌شود.  
	تغییر آنتروپی سیستم قالب یخ \lr{– }دریاچه را محاسبه کنید هنگامی که قالب یخ به تعادل حرارتی با دریاچه می‌رسد.  
	گرمای ویژه یخ $2220~\mathrm{J/(kg \cdot K)}$ است.  
	(راهنما: آیا قالب یخ دمای دریاچه را تحت تأثیر قرار می‌دهد؟
	\section{سوال چهارم}
	\
	 
	یک نمونه $2.0~\mathrm{mol}$ از یک گاز ایده‌آل تک‌اتمی فرآیند برگشت‌پذیر نشان داده شده در شکل زیر را طی می‌کند.  
	مقیاس محور عمودی با $T_s = 400.0~\mathrm{K}$ و مقیاس محور افقی با $S_s = 20.0~\mathrm{J/K}$ مشخص شده است.  
	
	(الف) چه مقدار انرژی به صورت گرما توسط گاز جذب می‌شود؟  
	(ب) تغییر انرژی داخلی گاز چقدر است؟  
	(ج) چه مقدار کار توسط گاز انجام می‌شود؟
	\begin{center}
		\includegraphics[width=0.5\textwidth]{../../../images/image-E_7/Image-1.png} \\[10pt] 
	\end{center}
	\section{سوال پنجم}
	
	یک موتور کارنو بین دماهای $235~^\circ\mathrm{C}$ و $115~^\circ\mathrm{C}$ کار می‌کند و در هر چرخه $6.30 \times 10^4~\mathrm{J}$ انرژی را از منبع دمای بالاتر جذب می‌کند.  
	
	(الف) بازده موتور چقدر است؟  
	(ب) این موتور در هر چرخه قادر به انجام چه مقدار کار است؟
	\section{سوال ششم}
	
	شکل زیریک موتور کارنو را نشان می‌دهد که بین دماهای $T_1 = 400~\mathrm{K}$ و $T_2 = 150~\mathrm{K}$ کار می‌کند و یک یخچال کارنو را که بین دماهای $T_3 = 325~\mathrm{K}$ و $T_4 = 225~\mathrm{K}$ کار می‌کند، به حرکت درمی‌آورد.  
	
	نسبت $Q_3 / Q_1$ چقدر است؟
		\begin{center}
		\includegraphics[width=0.5\textwidth]{../../../images/image-E_7/Image-2.png} \\[10pt] 
	\end{center}
	\section{سوال هفتم}
	 
	یک قالب یخ $8.0~\mathrm{g}$ با دمای $-10~^\circ\mathrm{C}$ در یک فلاسک حرارتی قرار داده می‌شود که حاوی $100~\mathrm{cm^3}$ آب با دمای $20~^\circ\mathrm{C}$ است.  
	
	تغییر آنتروپی سیستم قالب یخ \lr{–} آب پس از رسیدن به تعادل حرارتی چقدر است؟  
	گرمای ویژه یخ $2220~\mathrm{J/(kg \cdot K)}$ است.
	\section{سوال هشتم}
	
	انرژی می‌تواند به صورت گرما از آب خارج شود حتی در دمای پایین‌تر از نقطه انجماد عادی $(0.0~^\circ\mathrm{C}$ در فشار جوی) بدون اینکه آب یخ بزند؛ در این حالت آب به اصطلاح فوق‌سرد گفته می‌شود.  
	
	فرض کنید یک قطره آب $1.00~\mathrm{g}$ تا دمای هوای اطراف، که $-5.00~^\circ\mathrm{C}$ است، فوق‌سرد شود. سپس قطره ناگهان و غیرقابل برگشت یخ می‌زند و انرژی را به هوا به صورت گرما منتقل می‌کند.  
	
	تغییر آنتروپی برای این قطره چقدر است؟  
	(راهنما: از یک فرآیند برگشت‌پذیر سه مرحله‌ای استفاده کنید، انگار آب از نقطه انجماد معمولی عبور می‌کند.)  
	گرمای ویژه یخ $2220~\mathrm{J/(kg \cdot K)}$ است.
	\vspace{10pt}
	\\
	\textbf{موفق باشید.}
\end{document}