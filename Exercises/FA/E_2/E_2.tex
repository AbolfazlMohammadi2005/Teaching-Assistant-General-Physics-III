%---------------------------------------------------
% کلاس سند: article برای اسناد متنی، با اندازه کاغذ A4 و فونت پایه 16
\documentclass[a4paper,16pt]{article}

% بسته‌های ریاضی برای معادلات و نمادهای پیشرفته
\usepackage{amsmath, amssymb, amstext}

% بسته رنگ برای استفاده از رنگ در متن یا محیط‌ها
\usepackage{xcolor}
\usepackage{setspace}

% بسته گرافیک برای درج تصاویر
\usepackage{graphicx}

% تنظیمات حاشیه صفحه
\usepackage{geometry}

% بسته لینک‌دهی (هایپرلینک‌ها)
\usepackage{hyperref}

% بسته صفحه‌آرایی برای تنظیم هدر و فوتر
\usepackage{fancyhdr}


% بسته xepersian برای پشتیبانی از زبان فارسی با XeLaTeX
\usepackage{xepersian}

% تعیین فونت فارسی (فایل فونت باید موجود باشد یا در سیستم نصب باشد)
\settextfont{B Nazanin}

% تنظیم اندازه حاشیه‌ها از هر طرف برابر با 1 اینچ
\geometry{margin=1in}

% تنظیم سبک fancy برای سربرگ و پاورقی
\pagestyle{fancy}

% سمت چپ هدر: مشخص‌کننده سری تمرین و نام درس
\lhead{تمرین سری دو درس فیزیک\lr{ III} }

% وسط هدر: نام دانشکده
\chead{دانشکده فیزیک دانشگاه صنعتی خواجه نصیرالدین طوسی}

% سمت راست هدر: تاریخ روز به‌صورت خودکار
\rhead{\today}

% پایین صفحه: شماره صفحه
\cfoot{\thepage}

% شروع سند
\setstretch{1.3} % فاصله 1.3 برابر فاصله معمولی

\begin{document}
	
	% صفحه عنوان (بدون هدر و فوتر)
	\thispagestyle{empty}
	\begin{center}
		% لوگوی دانشگاه (باید فایل تصویر در مسیر پروژه باشد)
		
		\includegraphics[width=0.5\textwidth]{../../../images/image-E_2/K._N._Toosi_University_of_Technology.png} \\[10pt] 
		% عنوان سند
		\textbf{\LARGE عنوان:تمرین سری دو}\\[20pt]
		
		% نیم‌سال تحصیلی
		\textbf{\LARGE نیم‌سال تحصیلی:4041 }\\[10pt]
		
		% نام مدرس درس
		\textbf{\Large مدرس: دکتر امین نصیری‌راد}\\[10pt]
		
		% مبحث تمرین 
		\textbf{\Large  مبحث تمرین:شاره‌ها  }\\[10pt]
		
		% مهلت تحویل تمرین
		\textbf{\Large مهلت تحویل:۳آبان }
	\end{center}
	
	% صفحه جدید برای ادامه سند
	\newpage
	
	% ایجاد فهرست مطالب به‌صورت خودکار (براساس \section و \subsection و ...)
	\tableofcontents
	\newpage
	
	
	\section{سوال اول}
 دو جریان آب با هم ترکیب می‌شوند و یک رودخانه را تشکیل می‌دهند. یکی از جریان‌ها عرض \(8.2~\mathrm{m}\)، عمق \(3.4~\mathrm{m}\) و سرعت جریان \(2.3~\mathrm{m/s}\) دارد؛ جریان دیگر عرض \(6.8~\mathrm{m}\)، عمق \(3.2~\mathrm{m}\) و سرعت \(2.6~\mathrm{m/s}\) دارد. اگر رودخانه حاصل عرض \(10.5~\mathrm{m}\) و سرعت \(2.9~\mathrm{m/s}\) داشته باشد، عمق آن چقدر است؟
	
	
	\section{سوال دوم}
	آب با سرعت ثابت \(5.0~\mathrm{m/s}\) از یک زیرزمینِ پر از آب توسط پمپی از طریق شلنگی با شعاع \(1.0~\mathrm{cm}\) خارج می‌شود و از پنجره‌ای به ارتفاع \(3.0~\mathrm{m}\) بالاتر از سطح آب عبور می‌کند؛ توان پمپ چقدر است؟
	
	
	\section{سوال سوم}
	لوله پلاستیکیِ شکل زآبی که از یک لوله با قطر داخلی \(1.9~\mathrm{cm}\) عبور می‌کند، از طریق سه لوله با قطر \(1.3~\mathrm{cm}\) خارج می‌شود؛ \textbf{(الف)} اگر دبی‌ها در سه لوله کوچک‌تر به‌ترتیب \(26\)، \(19\) و \(11~\mathrm{L/min}\) باشند، دبی در لوله \(1.9~\mathrm{cm}\) چقدر است؟ \textbf{(ب)} نسبت سرعت جریان در لوله \(1.9~\mathrm{cm}\) به سرعت جریان در لوله‌ای که \(26~\mathrm{L/min}\) دبی دارد چقدر است؟
	
	\section{سوال چهارم}
	یک مخزن استوانه‌ای با قطر بزرگ تا عمق \(D=0.30~\mathrm{m}\) از آب پر شده است و سوراخی با سطح مقطع \(A=6.5~\mathrm{cm^2}\) در کف مخزن اجازه می‌دهد آب تخلیه شود؛ \textbf{(الف)} آهنگ تخلیه‌ی آب را بر حسب \(\mathrm{m^3/s}\) بیابید، \textbf{(ب)} در چه فاصله‌ای پایین‌تر از کف مخزن، سطح مقطع جریان آب برابر با نصف سطح مقطع سوراخ می‌شود؟
	
	\section{سوال پنجم}
\section*{لوله پیتوت}

یک لوله پیتوت (شکل ۱۴-۴۸) برای تعیین سرعت هواپیمای در حال حرکت استفاده می‌شود. این لوله شامل یک لوله بیرونی با چند سوراخ کوچک $B$ (چهار سوراخ نشان داده شده) است که اجازه می‌دهند هوا وارد لوله شود؛ این لوله به یکی از بازوهای یک لوله $U$ متصل است. بازوی دیگر لوله $U$ به سوراخ $A$ در انتهای جلویی دستگاه متصل است که به سمت مسیر حرکت هواپیما اشاره دارد. در نقطه $A$ هوا متوقف می‌شود، بنابراین $v_A = 0$؛ در نقطه $B$ سرعت هوا برابر با سرعت هواپیما $v$ است.

\textbf{(الف)} با استفاده از معادله برنولی نشان دهید که
\[
v = \sqrt{\frac{2 \rho g h}{\rho_\text{\lr{air}}}},
\]
که در آن $\rho$ چگالی مایع داخل لوله $U$ و $h$ اختلاف ارتفاع مایع در آن لوله است.

\textbf{(ب)} فرض کنید لوله حاوی الکل است و اختلاف سطح مایع $h = 26.0~\text{\lr{cm}}$ است. سرعت هواپیما نسبت به هوا چقدر است؟ چگالی هوا $\rho_\text{\lr{air}} = 1.03~\text{\lr{kg/m}}^3$ و چگالی الکل $\rho = 810~\text{\lr{kg/m}}^3$ داده شده است.

	\begin{center}
		\includegraphics[width=0.5\textwidth]{../../../images/image-E_2/Image-1.png} \\[10pt] 
	\end{center}
	
	\section{سوال ششم}
	
	
	یک سیال با چگالی $\rho = 900~\text{\lr{kg/m}}^3$ از یک لوله افقی عبور می‌کند که سطح مقطع آن در ناحیه $A$ برابر $A_A = 1.90 \times 10^{-2}~\text{\lr{m}}^2$ و در ناحیه $B$ برابر $A_B = 9.50 \times 10^{-2}~\text{\lr{m}}^2$ است. اختلاف فشار بین دو ناحیه برابر $\Delta P = 7.20 \times 10^3~\text{\lr{Pa}}$ است.  
	
	\textbf{(الف)} دبی حجمی جریان را بیابید.  
	\textbf{(ب)} دبی جرمی جریان را بیابید.
	\section{سوال هفتم}
	
	\textbf{(الف)} برای آب دریا با چگالی $\rho = 1.03~\text{\lr{g/cm}}^3$، وزن آب روی زیردریایی در عمق $
	255~\text{\lr{m}}$ را بیابید، اگر سطح مقطع افقی بدنه برابر $A = 2200.0~\text{\lr{m}}^2$ باشد.  
	
	\textbf{(ب)} فشار آب را در این عمق به واحد اتمسفر برای یک غواص محاسبه کنید.
	
	\vspace{10pt}
	\textbf{موفق باشید.}
\end{document}