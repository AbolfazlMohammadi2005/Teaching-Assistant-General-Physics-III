%---------------------------------------------------
% کلاس سند: article برای اسناد متنی، با اندازه کاغذ A4 و فونت پایه 16
\documentclass[a4paper,16pt]{article}

% بسته‌های ریاضی برای معادلات و نمادهای پیشرفته
\usepackage{amsmath, amssymb, amstext}

% بسته رنگ برای استفاده از رنگ در متن یا محیط‌ها
\usepackage{xcolor}
\usepackage{setspace}

% بسته گرافیک برای درج تصاویر
\usepackage{graphicx}

% تنظیمات حاشیه صفحه
\usepackage{geometry}

% بسته لینک‌دهی (هایپرلینک‌ها)
\usepackage{hyperref}

% بسته صفحه‌آرایی برای تنظیم هدر و فوتر
\usepackage{fancyhdr}


% بسته xepersian برای پشتیبانی از زبان فارسی با XeLaTeX
\usepackage{xepersian}

% تعیین فونت فارسی (فایل فونت باید موجود باشد یا در سیستم نصب باشد)
\settextfont{B Nazanin}

% تنظیم اندازه حاشیه‌ها از هر طرف برابر با 1 اینچ
\geometry{margin=1in}

% تنظیم سبک fancy برای سربرگ و پاورقی
\pagestyle{fancy}

% سمت چپ هدر: مشخص‌کننده سری تمرین و نام درس
\lhead{تمرین سری شش درس فیزیک\lr{ III} }

% وسط هدر: نام دانشکده
\chead{دانشکده فیزیک دانشگاه صنعتی خواجه نصیرالدین طوسی}

% سمت راست هدر: تاریخ روز به‌صورت خودکار
\rhead{\today}

% پایین صفحه: شماره صفحه
\cfoot{\thepage}

% شروع سند
\setstretch{1.3} % فاصله 1.3 برابر فاصله معمولی

\begin{document}
	
	% صفحه عنوان (بدون هدر و فوتر)
	\thispagestyle{empty}
	\begin{center}
		% لوگوی دانشگاه (باید فایل تصویر در مسیر پروژه باشد)
		
		\includegraphics[width=0.5\textwidth]{../../../images/image-E_6/K._N._Toosi_University_of_Technology.png} \\[10pt] 
		% عنوان سند
		\textbf{\LARGE عنوان:تمرین سری شش}\\[20pt]
		
		% نیم‌سال تحصیلی
		\textbf{\LARGE نیم‌سال تحصیلی:4041 }\\[10pt]
		
		% نام مدرس درس
		\textbf{\Large مدرس: دکتر امین نصیری‌راد}\\[10pt]
		
		% مبحث تمرین 
		\textbf{\Large  مبحث تمرین:نظریه جنبشی گازها  }\\[10pt]
		
		% مهلت تحویل تمرین
		\textbf{\Large مهلت تحویل:۱۸آذر }
	\end{center}
	
	% صفحه جدید برای ادامه سند
	\newpage
	
	% ایجاد فهرست مطالب به‌صورت خودکار (براساس \section و \subsection و ...)
	\tableofcontents
	\newpage
	
	
	\section{سوال اول}
	
	 
	یک مول گاز ایده‌آل دواتمی از نقطه \lr{a} به \lr{c} در طول مسیر مورب در شکل زیرمی‌رود.  
	مقیاس محور عمودی با $p_{ab} = 5.0~\mathrm{kPa}$ و $p_c = 2.0~\mathrm{kPa}$ تنظیم شده است، و مقیاس محور افقی با $V_{bc} = 4.0~\mathrm{m^3}$ و $V_a = 2.0~\mathrm{m^3}$ مشخص شده است.  
	
	(الف) در طول این گذار، تغییر انرژی داخلی گاز چقدر است؟  
	(ب) چه مقدار انرژی به صورت گرما به گاز اضافه می‌شود؟  
	(ج) اگر گاز از \lr{a} به \lr{c }از مسیر غیرمستقیم \lr{abc }عبور کند، چه مقدار گرما لازم است؟
		\begin{center}
		\includegraphics[width=0.5\textwidth]{../../../images/image-E_6/Image-1.png} \\[10pt] 
	\end{center}
	\section{سوال دوم}
	
	یک ظرف شامل مخلوطی از سه گاز غیرواکنشی است:  
	$2.40~\mathrm{mol}$ گاز ۱ با $C_{V1} = 12.0~\mathrm{J/(mol\cdot K)}$،  
	$1.50~\mathrm{mol}$ گاز ۲ با $C_{V2} = 12.8~\mathrm{J/(mol\cdot K)}$، و  
	$3.20~\mathrm{mol}$ گاز ۳ با $C_{V3} = 20.0~\mathrm{J/(mol\cdot K)}$.  
	
	$C_V$ مخلوط چقدر است؟	
	\section{سوال سوم}
	
 
فرض کنید $12.0~\mathrm{g}$ گاز اکسیژن $(\mathrm{O_2})$ در فشار جوی ثابت از $25.0~^\circ\mathrm{C}$ تا $125~^\circ\mathrm{C}$ گرم می‌شود.  

(الف) چند مول اکسیژن وجود دارد؟ (برای جرم مولی به جدول 19-1 مراجعه کنید.)  
(ب) چه مقدار انرژی به صورت گرما به اکسیژن منتقل می‌شود؟ (مولکول‌ها می‌چرخند اما نوسان نمی‌کنند.)  
(ج) چه کسری از گرما صرف افزایش انرژی داخلی اکسیژن می‌شود؟
	\section{سوال چهارم}
	\
	یک گاز در دمای $310~\mathrm{K}$ و فشار $1.2~\mathrm{atm}$ حجم $4.3~\mathrm{L}$ را اشغال می‌کند. این گاز به صورت آدیاباتیک تا حجم $0.76~\mathrm{L}$ فشرده می‌شود.  
	
	(الف) فشار نهایی چقدر است؟  
	(ب) دمای نهایی چقدر است، با فرض اینکه گاز ایده‌آل است و $\gamma = 1.4$ باشد.
	\section{سوال پنجم}
	

شکل زیر یک چرخه را نشان می‌دهد که توسط $1.00~\mathrm{mol}$ گاز ایده‌آل تک‌اتمی طی می‌شود. دماها عبارتند از $T_1 = 300~\mathrm{K}$، $T_2 = 600~\mathrm{K}$، و $T_3 = 455~\mathrm{K}$. فشار اولیه در نقطه ۱ برابر $1.00~\mathrm{atm}~(= 1.013 \times 10^5~\mathrm{Pa})$ است.  

برای گذار $1 \rightarrow 2$:  
(الف) گرما $Q$  
(ب) تغییر انرژی داخلی $\Delta E_\mathrm{int}$  
(ج) کار انجام شده $W$  

برای گذار $2 \rightarrow 3$:  
(د) $Q$  
(ه) $\Delta E_\mathrm{int}$  
(و) $W$  

برای گذار $3 \rightarrow 1$:  
(ز) $Q$  
(ح) $\Delta E_\mathrm{int}$  
(ط) $W$  

برای کل چرخه:  
(ی) $Q$  
(ک) $\Delta E_\mathrm{int}$  
(ل) $W$  

حجم و فشار نقاط دیگر:  
(م) حجم و (ن) فشار در نقطه ۲  
(س) حجم و (ع) فشار در نقطه ۳
\begin{center}
	\includegraphics[width=0.5\textwidth]{../../../images/image-E_6/Image-2.png} \\[10pt] 
\end{center}
	\section{سوال ششم}
	 
	در طول یک تراکم در فشار ثابت $250~\mathrm{Pa}$، حجم یک گاز ایده‌آل از $0.80~\mathrm{m^3}$ به $0.20~\mathrm{m^3}$ کاهش می‌یابد. دمای اولیه $360~\mathrm{K}$ است و گاز $210~\mathrm{J}$ گرما از دست می‌دهد.  
	
	(الف) تغییر انرژی داخلی گاز چقدر است؟  
	(ب) دمای نهایی گاز چقدر است؟
	\section{سوال هفتم}
	
یک گاز ایده‌آل از طریق یک چرخه کامل در سه مرحله طی می‌شود:  
توسعه آدیاباتیک با کار $125~\mathrm{J}$،  
انقباض ایزوترمیک در دمای $325~\mathrm{K}$، و  
افزایش فشار در حجم ثابت.  

(الف) نمودار $p$\lr{–}$V$ برای سه مرحله را رسم کنید.  
(ب) چه مقدار انرژی به صورت گرما در مرحله ۳ منتقل می‌شود؟  
(ج) آیا این انرژی به گاز منتقل می‌شود یا از گاز گرفته می‌شود؟
	\vspace{10pt}
	\\
	\textbf{موفق باشید.}
\end{document}