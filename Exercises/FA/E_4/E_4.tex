%---------------------------------------------------
% کلاس سند: article برای اسناد متنی، با اندازه کاغذ A4 و فونت پایه 16
\documentclass[a4paper,16pt]{article}

% بسته‌های ریاضی برای معادلات و نمادهای پیشرفته
\usepackage{amsmath, amssymb, amstext}

% بسته رنگ برای استفاده از رنگ در متن یا محیط‌ها
\usepackage{xcolor}
\usepackage{setspace}

% بسته گرافیک برای درج تصاویر
\usepackage{graphicx}

% تنظیمات حاشیه صفحه
\usepackage{geometry}

% بسته لینک‌دهی (هایپرلینک‌ها)
\usepackage{hyperref}

% بسته صفحه‌آرایی برای تنظیم هدر و فوتر
\usepackage{fancyhdr}


% بسته xepersian برای پشتیبانی از زبان فارسی با XeLaTeX
\usepackage{xepersian}

% تعیین فونت فارسی (فایل فونت باید موجود باشد یا در سیستم نصب باشد)
\settextfont{B Nazanin}

% تنظیم اندازه حاشیه‌ها از هر طرف برابر با 1 اینچ
\geometry{margin=1in}

% تنظیم سبک fancy برای سربرگ و پاورقی
\pagestyle{fancy}

% سمت چپ هدر: مشخص‌کننده سری تمرین و نام درس
\lhead{تمرین سری چهار درس فیزیک\lr{ III} }

% وسط هدر: نام دانشکده
\chead{دانشکده فیزیک دانشگاه صنعتی خواجه نصیرالدین طوسی}

% سمت راست هدر: تاریخ روز به‌صورت خودکار
\rhead{\today}

% پایین صفحه: شماره صفحه
\cfoot{\thepage}

% شروع سند
\setstretch{1.3} % فاصله 1.3 برابر فاصله معمولی

\begin{document}
	
	% صفحه عنوان (بدون هدر و فوتر)
	\thispagestyle{empty}
	\begin{center}
		% لوگوی دانشگاه (باید فایل تصویر در مسیر پروژه باشد)
		
		\includegraphics[width=0.5\textwidth]{../../../images/image-E_4/K._N._Toosi_University_of_Technology.png} \\[10pt] 
		% عنوان سند
		\textbf{\LARGE عنوان:تمرین سری چهار}\\[20pt]
		
		% نیم‌سال تحصیلی
		\textbf{\LARGE نیم‌سال تحصیلی:4041 }\\[10pt]
		
		% نام مدرس درس
		\textbf{\Large مدرس: دکتر امین نصیری‌راد}\\[10pt]
		
		% مبحث تمرین 
		\textbf{\Large  مبحث تمرین:دما و گرما  }\\[10pt]
		
		% مهلت تحویل تمرین
		\textbf{\Large مهلت تحویل:۲۴آبان }
	\end{center}
	
	% صفحه جدید برای ادامه سند
	\newpage
	
	% ایجاد فهرست مطالب به‌صورت خودکار (براساس \section و \subsection و ...)
	\tableofcontents
	\newpage
	
	
	\section{سوال اول}
	
	در شکل زیر نمونه‌ای از گاز از حجم $V_0$ به $4.0\,V_0$ منبسط می‌شود در حالی که فشار آن از $p_0$ به $p_0/4.0$ کاهش می‌یابد. اگر $V_0 = 1.0~\mathrm{m^3}$ و $p_0 = 40~\mathrm{Pa}$ باشد، چه مقدار کار توسط گاز انجام می‌شود اگر فشار آن با حجم از طریق مسیرهای زیر تغییر کند:  
	
	(الف) مسیر \lr{A  }
	(ب) مسیر \lr{B } 
	(ج) مسیر \lr{C}
	
	\begin{center}
		\includegraphics[width=0.5\textwidth]{../../../images/image-E_4/Image-1.png} \\[10pt] 
	\end{center}
	\section{سوال دوم}
	
	شکل زیریک چرخه بسته برای یک گاز را نشان می‌دهد (شکل به مقیاس نیست). تغییر انرژی داخلی گاز هنگامی که از نقطه \lr{a} به \lr{c} از مسیر \lr{abc} حرکت می‌کند، $-200~\mathrm{J}$ است.  
	\\
	(الف) هنگامی که گاز از \lr{c} به \lr{d }حرکت می‌کند، $180~\mathrm{J}$ گرما باید به آن منتقل شود.
	\\  
	(ب) انتقال اضافی $80~\mathrm{J}$ گرما لازم است هنگامی که گاز از \lr{d }به \lr{a }حرکت می‌کند.
	\\  
	(ج) چه مقدار کار بر روی گاز انجام می‌شود هنگامی که از \lr{c }به\lr{ d }حرکت می‌کند؟
	\begin{center}
		\includegraphics[width=0.5\textwidth]{../../../images/image-E_4/Image-2.png} \\[10pt] 
	\end{center}
	\section{سوال سوم}
	 
	یک میله استوانه‌ای مسی به طول $1.2~\mathrm{m}$ و سطح مقطع $4.8~\mathrm{cm^2}$ در طول جانبی خود عایق شده است. دو سر میله تحت اختلاف دمای $100~^\circ\mathrm{C}$ نگه داشته شده‌اند، به طوری که یک سر در مخلوط آب و یخ و سر دیگر در مخلوط آب جوش و بخار قرار دارد.  
	
	(الف) نرخ انتقال انرژی توسط میله چقدر است؟  
	(ب) یخ با چه سرعتی ذوب می‌شود؟
	
	\section{سوال چهارم}
	
	 
	یک استوانه جامد با شعاع $r_1 = 2.5~\mathrm{cm}$، طول $h_1 = 5.0~\mathrm{cm}$، ضریب تابش $\epsilon = 0.85$ و دما $30~^\circ\mathrm{C}$ در محیطی با دمای $50~^\circ\mathrm{C}$ آویزان شده است. 
	\\ 
	
(الف) نرخ انتقال خالص تابشی استوانه $P_1$ چقدر است؟
	\\  
(ب) اگر استوانه کشیده شود تا شعاع آن به $r_2 = 0.50~\mathrm{cm}$ برسد، نرخ انتقال خالص تابشی آن $P_2$ می‌شود. نسبت $P_2 / P_1$ چقدر است؟
	
	\section{سوال پنجم}
	 
	روی یک برکه کم‌عمق یخ بسته است و حالت پایا برقرار شده است، به طوری که دمای هوا بالای یخ $-5.0~^\circ\mathrm{C}$ و دمای کف برکه $4.0~^\circ\mathrm{C}$ است. اگر عمق کل یخ و آب $1.4~\mathrm{m}$ باشد، ضخامت یخ چقدر است؟  
	(فرض کنید ضریب هدایت حرارتی یخ و آب به ترتیب $0.40$ و $0.12~\mathrm{cal/(m\cdot ^\circ C \cdot s)}$ باشد.)
	\section{سوال ششم}
	
	شکل زیر یک چرخه بسته برای یک گاز را نشان می‌دهد.  
	
	(الف) از \lr{c } به \lr{b، } مقدار $40~\mathrm{J}$ از گاز به صورت گرما منتقل می‌شود.
	\\  
	(ب) از \lr{b} به \lr{a، }مقدار $130~\mathrm{J}$ از گاز به صورت گرما منتقل می‌شود و مقدار کار انجام شده توسط گاز $80~\mathrm{J}$ است.
	\\  
	(ج) از \lr{a} به \lr{c،} مقدار $400~\mathrm{J}$ به گاز به صورت گرما منتقل می‌شود.
	\\  
	کار انجام شده توسط گاز از \lr{a} به \lr{c} چقدر است؟  
	\\
	(راهنما: لازم است برای داده‌های داده شده علامت‌های مثبت و منفی را مشخص کنید.)
		\begin{center}
		\includegraphics[width=0.5\textwidth]{../../../images/image-E_4/Image-3.png} \\[10pt] 
	\end{center}
	\section{سوال هفتم}
	
	یک میله مسی، یک میله آلومینیومی و یک میله برنجی، هر کدام به طول $6.00~\mathrm{m}$ و قطر $1.00~\mathrm{cm}$، به صورت پشت سر هم قرار داده شده‌اند، به طوری که میله آلومینیومی بین دو میله دیگر قرار دارد.  
	سر آزاد میله مسی در نقطه جوش آب و سر آزاد میله برنجی در نقطه انجماد آب نگه داشته شده است.  
	
	(الف) دمای پایا در محل اتصال مس و آلومینیوم چقدر است؟  
	(ب) دمای پایا در محل اتصال آلومینیوم و برنج چقدر است؟
	
	
	
	\textbf{موفق باشید.}
\end{document}