%---------------------------------------------------
% کلاس سند: article برای اسناد متنی، با اندازه کاغذ A4 و فونت پایه 16
\documentclass[a4paper,16pt]{article}

% بسته‌های ریاضی برای معادلات و نمادهای پیشرفته
\usepackage{amsmath, amssymb, amstext}

% بسته رنگ برای استفاده از رنگ در متن یا محیط‌ها
\usepackage{xcolor}
\usepackage{setspace}

% بسته گرافیک برای درج تصاویر
\usepackage{graphicx}

% تنظیمات حاشیه صفحه
\usepackage{geometry}

% بسته لینک‌دهی (هایپرلینک‌ها)
\usepackage{hyperref}

% بسته صفحه‌آرایی برای تنظیم هدر و فوتر
\usepackage{fancyhdr}


% بسته xepersian برای پشتیبانی از زبان فارسی با XeLaTeX
\usepackage{xepersian}

% تعیین فونت فارسی (فایل فونت باید موجود باشد یا در سیستم نصب باشد)
\settextfont{B Nazanin}

% تنظیم اندازه حاشیه‌ها از هر طرف برابر با 1 اینچ
\geometry{margin=1in}

% تنظیم سبک fancy برای سربرگ و پاورقی
\pagestyle{fancy}

% سمت چپ هدر: مشخص‌کننده سری تمرین و نام درس
\lhead{تمرین سری یک درس فیزیک\lr{ III} }

% وسط هدر: نام دانشکده
\chead{دانشکده فیزیک دانشگاه صنعتی خواجه نصیرالدین طوسی}

% سمت راست هدر: تاریخ روز به‌صورت خودکار
\rhead{\today}

% پایین صفحه: شماره صفحه
\cfoot{\thepage}

% شروع سند
\setstretch{1.3} % فاصله 1.3 برابر فاصله معمولی

\begin{document}
	
	% صفحه عنوان (بدون هدر و فوتر)
	\thispagestyle{empty}
	\begin{center}
		% لوگوی دانشگاه (باید فایل تصویر در مسیر پروژه باشد)
		
		\includegraphics[width=0.5\textwidth]{../../../images/image-E_1/K._N._Toosi_University_of_Technology.png} \\[10pt] 
		% عنوان سند
		\textbf{\LARGE عنوان:تمرین سری‌یک}\\[20pt]
		
		% نیم‌سال تحصیلی
		\textbf{\LARGE نیم‌سال تحصیلی:4041 }\\[10pt]
		
		% نام مدرس درس
		\textbf{\Large مدرس: دکتر امین نصیری‌راد}\\[10pt]
		
		% مبحث تمرین 
		\textbf{\Large  مبحث تمرین:شاره‌ها  }\\[10pt]
		
		% مهلت تحویل تمرین
		\textbf{\Large مهلت تحویل:۲۶مهر }
	\end{center}
	
	% صفحه جدید برای ادامه سند
	\newpage
	
	% ایجاد فهرست مطالب به‌صورت خودکار (براساس \section و \subsection و ...)
	\tableofcontents
	\newpage
	
	
	\section{سوال اول}
	یک محفظه نیمه‌خلأ و کاملاً بسته دارای درپوشی محکم با سطح \(77~\mathrm{m^2}\) و جرم ناچیز است. اگر نیروی لازم برای برداشتن درپوش \(480~\mathrm{N}\) و فشار جو \(1.0 \times 10^5~\mathrm{Pa}\) باشد، فشار هوای داخلی محفظه چقدر است؟
	
	
	\section{سوال دوم}
	 در سال 1654 اتو فون گریکه، مخترع پمپ هوا، نمایشگاهی در حضور نجیب‌زادگان امپراتوری مقدس روم انجام داد که در آن دو تیم ۸ اسب نتوانستند دو نیمکره برنجی تخلیه‌شده را از هم جدا کنند.
	 \\  
	\textbf{(الف)} با فرض اینکه نیمکره‌ها دیواره‌ای نازک و محکم دارند و شعاع \(R\) در شکل زیر هم شعاع داخلی و هم خارجی محسوب می‌شود، نشان دهید نیروی لازم برای جدا کردن نیمکره‌ها \(F = \pi R^2 \Delta p\) است، که \(\Delta p\) اختلاف فشار داخل و خارج نیمکره است.
	\\  
	\textbf{(ب)} با فرض \(R = 30~\mathrm{cm}\)، فشار داخلی \(0.10~\mathrm{atm}\) و فشار خارجی \(1.00~\mathrm{atm}\)، مقدار نیرویی که تیم‌های اسب باید اعمال می‌کردند را بیابید
	\\
	\textbf{(ج)} توضیح دهید چرا یک تیم اسب می‌توانست به همان اندازه مؤثر نشان دهد اگر نیمکره‌ها به یک دیوار محکم متصل می‌شدند.
	\begin{center}
		\includegraphics[width=0.5\textwidth]{../../../images/image-E_1/Image-1.png} \\[10pt] 
	\end{center}
	
	\section{سوال سوم}
	لوله پلاستیکیِ شکل زیر سطح مقطع \(5.00~\mathrm{cm^2}\) دارد. لوله ابتدا با آب پر می‌شود تا بازوی کوتاه آن با طول \(d = 0.800~\mathrm{m}\) کاملاً پر شود؛ سپس بازوی کوتاه مهر و موم می‌شود و به‌تدریج آب بیشتری در بازوی بلند ریخته می‌شود. اگر این درپوش زمانی جدا شود که نیروی وارد بر آن از \(9.80~\mathrm{N}\) بیشتر شود، ارتفاع کل آب در بازوی بلند که در آستانه جدا شدن درپوش قرار می‌دهد چقدر است؟
	
	\begin{center}
	\includegraphics[width=0.5\textwidth]{../../../images/image-E_1/Image-2.png} \\[10pt] 
\end{center}
	
	\section{سوال چهارم}
	یک آکواریوم بزرگ با ارتفاع \(5.00~\mathrm{m}\) تا عمق \(2.00~\mathrm{m}\) با آب شیرین پر شده است. یکی از دیواره‌های آکواریوم از پلاستیک ضخیم با عرض \(8.00~\mathrm{m}\) تشکیل شده است. اگر آکواریوم تا عمق \(4.00~\mathrm{m}\) پر شود، نیروی کل وارد بر آن دیواره چه مقدار افزایش می‌یابد؟
	
	\section{سوال پنجم}
	در شکل زیریک فنر با ثابت فنر \(3.00 \times 10^4~\mathrm{N/m}\) بین یک تیر صلب و پیستون خروجی یک اهرم هیدرولیکی قرار دارد. یک ظرف خالی با جرم ناچیز روی پیستون ورودی قرار گرفته است. مساحت پیستون ورودی \(A_i\) و مساحت پیستون خروجی \(18.0A_i\) است. در ابتدا فنر در طول طبیعی خود قرار دارد. چه مقدار جرم شن باید به‌آرامی داخل ظرف ریخته شود تا فنر به اندازه \(5.00~\mathrm{cm}\) فشرده شود؟
	\begin{center}
		\includegraphics[width=0.5\textwidth]{../../../images/image-E_1/Image-3.png} \\[10pt] 
	\end{center}
	
	\section{سوال ششم}
	
	شکل زیر یک گلوله آهنی را نشان می‌دهد که با نخی با جرم ناچیز از یک استوانه قائم آویزان شده است؛ استوانه به‌طور جزئی در آب شناور است. استوانه ارتفاع \(6.00~\mathrm{cm}\)، مساحت سطح بالا و پایین \(12.0~\mathrm{cm^2}\)، و چگالی \(0.30~\mathrm{g/cm^3}\) دارد و \(2.00~\mathrm{cm}\) از ارتفاع آن بالای سطح آب قرار گرفته است. شعاع گلوله آهنی چقدر است؟
	\begin{center}
		\includegraphics[width=0.5\textwidth]{../../../images/image-E_1/Image-4.png} \\[10pt] 
	\end{center}
	\section{سوال هفتم}
	 یک جسم \(5.00~\mathrm{kg}\) در حالی که کاملاً درون یک مایع غوطه‌ور است، از حالت سکون رها می‌شود. جرم مایع جابه‌جا‌شده توسط جسم غوطه‌ور \(3.00~\mathrm{kg}\) است. با فرض حرکت آزاد جسم و ناچیز بودن نیروی پسا از طرف مایع، جسم در مدت \(0.200~\mathrm{s}\) چه مسافتی و در چه جهتی جابه‌جا می‌شود؟
	
	\vspace{10pt}
	\textbf{موفق باشید.}
\end{document}